\documentclass[11pt, oneside]{article}
\usepackage[margin=1in]{geometry}
\geometry{letterpaper}
\usepackage{graphicx}
\usepackage{amssymb}
\usepackage[parfill]{parskip}
\usepackage{amssymb}
\usepackage{amsmath}
\usepackage{listings}
\usepackage{color}
\usepackage{standalone}
\usepackage{gensymb}
\usepackage{tikz}
\usetikzlibrary{matrix,chains,positioning,decorations.pathreplacing,arrows}
\usepackage{wrapfig}
\usepackage{epstopdf}

\graphicspath{ {images/} }

\def\layersep{2.5cm}

\sloppy
\definecolor{lightgray}{gray}{0.5}
\setlength{\parindent}{0pt}
\definecolor{dkgreen}{rgb}{0,0.6,0}
\definecolor{gray}{rgb}{0.5,0.5,0.5}
\definecolor{mauve}{rgb}{0.58,0,0.82}

\lstset{frame=tb,
  language=Matlab,
  aboveskip=3mm,
  belowskip=3mm,
  showstringspaces=false,
  columns=flexible,
  basicstyle={\small\ttfamily},
  numbers=none,
  numberstyle=\tiny\color{gray},
  keywordstyle=\color{blue},
  commentstyle=\color{dkgreen},
  stringstyle=\color{mauve},
  breaklines=true,
  breakatwhitespace=true,
  tabsize=3
}

\title{Neuro 120 Homework 4: Short and Long Term Memory}
\author{William Schmitt and Will Drew}
\date{Due: Thursday 15 November 2018}

\begin{document}
\maketitle

\section{Short Term Memory}

\subsection{RNN Dynamics with No Recurrent Connections}

We build the RNN with the following code:
\lstinputlisting{shortterm_mem.m}
This code produces Figure \ref{fig:RNNnoR}, which we can see tells us that the activity in the network has died out at $t = 2.8$s.

\subsection{RNN with Autapses}

We comment out line 16 and uncomment line 18 and 19 of the code attached above to run the network 

\begin{figure}[ht!]
\includegraphics[width=1\textwidth]{RNNnoR.eps}
\caption{The dynamics of a RNN with no recurrent connections.}
\label{fig:RNNnoR}
\end{document}
